Hyperspectral imaging has shown promise in clinical applications as a non-invasive imaging modality to provide spatially resolved parameters of interest. Whilst many demonstrations have been shown of hyperspectral cameras tested in surgical environments, there are limited demonstrations of quantitative hyperspectral data extraction in this setting due to constraints in traditional white balancing techniques. There have also been numerous demonstrations of oxygen saturation ($StO_2$) extraction from clinical hyperspectral data, however many of these models have not been validated using measurements with known ground-truth.

In this work we develop a novel synthetic white balancing technique which can be integrated well into surgical settings to allow quantitative snapshot hyperspectral data to be acquired as well as allowing a simplified approach to obtaining good quality relative data in this environment. This is followed by a comparison of three prominent analytical tissue models for semi-infinite homogeneous tissue. These are compared using simulated Monte Carlo data as well as gelatin-based tissue phantoms measured with an integrating sphere spectrophotometer. One of these is tested as a two-layer model with further Monte Carlo simulations and NIST skin data to determine it’s efficacy. The single-layer models demonstrate similar results for StO2 when using quantitative or relative data and so is adapted to hyperspectral data. This is tested initially on Monte Carlo simulations adapted to mimic a snapshot hyperspectral camera and then on measured hyperspectral data of the same gelatin-based phantoms and from neurosurgical cases.