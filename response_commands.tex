% to change the colour of text
\usepackage{xcolor}
% to strike through text
\usepackage[normalem]{ulem}

% to help with quoting added material in the revised paper 
\usepackage{quoting}
\newcommand{\say}[1]{%
\begin{quoting}
#1
\end{quoting}
}

\newcommand{\sayold}[1]{%
\say{{\color{olive}#1}}
}

\newcommand{\saydel}[1]{%
\say{{\color{darkgray}\sout{#1}}}
}

\newcommand{\saymod}[1]{%
\say{{\color{blue}#1}}
}

\newcommand{\saynew}[1]{%
\say{{\color{orange}#1}}
}

\newcommand{\reviewcomment}[3]{%
\vspace{4mm}
\textbf{C[$R_{#1}C_{#2}$]:} {\it #3}
\vspace{2mm}
}

\newcommand{\response}[1]{%
{\textbf{R:}} {#1}
}

% Change the default figure/table numbering to avoid confusion with main manuscript
\renewcommand{\figurename}{Fig.}
\renewcommand{\thefigure}{L\arabic{figure}}
\renewcommand{\thetable}{L\arabic{table}} 
\newcommand{\figref}[1]{\figurename~\ref{#1}}
\newcommand{\tabref}[1]{\tablename~\ref{#1}}